\documentclass[12pt]{article}
\usepackage[utf8]{inputenc}
\usepackage{float}
\usepackage{amsmath}


\usepackage[hmargin=3cm,vmargin=6.0cm]{geometry}
%\topmargin=0cm
\topmargin=-2cm
\addtolength{\textheight}{6.5cm}
\addtolength{\textwidth}{2.0cm}
%\setlength{\leftmargin}{-5cm}
\setlength{\oddsidemargin}{0.0cm}
\setlength{\evensidemargin}{0.0cm}

%misc libraries goes here
\usepackage{amsmath}

\begin{document}

\section*{Student Information } 
%Write your full name and id number between the colon and newline
%Put one empty space character after colon and before newline
Full Name : Deniz Rasim Uluğ  \\
Id Number : 2172088 \\

% Write your answers below the section tags
\section*{Answer 1}
Let's first prove the theorem for case $n=1$.\\
Base Step: $\sum_{j=1}^1 j(j+1)(j+2)\cdots (j+k-1)=1\cdot 2\cdot 3\cdots k, \cfrac{1\cdot 2\cdots k\cdot 1+k}{k+1}=1\cdot 2\cdot 3\cdots k$\\
Inductive Step: Let's assume:
\begin{center}
$\sum_{j=1}^n j(j+1)(j+2)\cdots (j+k-1)=\cfrac{n\cdot (n+1)\cdot (n+2)\cdots (n+k)}{(k+1)}$
\end{center}
And try to prove:
\begin{center}
$\sum_{j=1}^{n+1} j(j+1)(j+2)\cdots (j+k-1)=\cfrac{(n+1)\cdot (n+2)\cdot (n+3)\cdots (n+1+k)}{(k+1)}$
\end{center}
Let's start:
\begin{center}
$\sum_{j=1}^{n+1} j(j+1)(j+2)\cdots (j+k-1)=\sum_{j=1}^n j(j+1)(j+2)\cdots (j+k-1)+\big [(n+1)\cdot (n+2)\cdots (n+k) \big ]$\\
$=\cfrac{n\cdot (n+1)\cdot (n+2)\cdots (n+k)}{(k+1)}+\big [(n+1)\cdot (n+2)\cdots (n+k) \big ]$\\
$=\cfrac{n\cdot (n+1)\cdot (n+2)\cdots (n+k)}{(k+1)}+\big [\cfrac{(n+1)\cdot (n+2)\cdots (n+k)\cdot (k+1)}{(k+1)} \big ]$\\
$=\cfrac{(n+1)\cdot (n+2)\cdot (n+3)\cdots (n+1+k)}{(k+1)}$
\end{center}
Thus our proof is concluded.

\section*{Answer 2}
Let's define $P(n)$ to be the truth value of $H_n\leq 7^n$ and prove $P(n)$ for $n=0,1,2$\\
Base Step:
\begin{center}
$P(0)=> 1\leq 7^0=1$\\
$P(1)=> 3\leq 7^1=7$\\
$P(2)=> 5\leq 7^2=49$\\
\end{center}
Inductive Step: Let's assume $P(k)$ to be true $\forall k<n$ and try to prove $P(n)$.
\begin{center}
$H_n=5H_{n-1}+5H_{n-2}+63H_{n-3}$\\
$\leq 5\cdot 7^{n-1}+5\cdot 7^{n-2}+63\cdot 7^{n-3}$\\
$=245\cdot 7^{n-3}+ 35\cdot 7^{n-3}+63\cdot 7^{n-3}$\\
$=343\cdot 7^{n-3}=7^n$
\end{center}
We found $H_n\leq 7^n$ which concludes our proof.

\section*{Answer 3}
\textbf{a)}\\
We can find all possible collections of books and subtract from this the possible collections where all books are \textit{Signals and Systems}:
\begin{center}
$C(12,4)-C(7,4)=495-35=460$
\end{center}
\textbf{b)}\\
Conversely, we can find all possible collection of books and subtract from this both the possible collections where all books are \textit{Signals and Systems} and all are \textit{Discrete Mathematics}:
\begin{center}
$C(12,4)-C(7,4)-C(5,4)=495-35-5=455$
\end{center}
\section*{Answer 4}
Let's try to find a recursive formula $F(n)$ where $n$ is the length of such strings. By the sum rule, we can find such strings as the sum of those which end with $2$ and those which end with $3$.\\
For the former, we can simply add $2$ to the end of all strings in $F(n-1)$, which is also the number of such strings.\\
For the latter, we can again use sum rule to divide such strings in two, those ending with $33$ and those ending with $23$.\\
For the former, we can again add $33$ to the end of all strings in $F(n-2)$, since those strings already have even number of 3's and we are adding another two 3, this won't change their properties of having even number of 3's. For the latter tough we cannot do the same since we are adding an even number of 3's, namely just one. Instead tough we can add $23$ to end of all strings that have odd number of 3's(and has length $n-2$), and it is easy to see the count of such strings is given by $2^{n-2}-F(n-2)$.\\
So we can conclude:
\begin{center}
$F(n)=F(n-1)+F(n-2)+[2^{n-2}-F(n-2)]=F(n-1)+2^{n-2}$
\end{center}
And we can take our base step to be $F(1)=1$ since we can only write one of such string, namely the string "2".
\section*{Answer 5}
We can solve the relation like described in Theorem 2 of chapter 8.3 in book \textit{Discrete Mathematics and Its Applications} (Rosen). Reader can refer to book for proof. Now let's try to solve the relation, starting with finding characteristic roots of it.
\begin{center}
$a_n=4a_{n-1}+a_{n-2}-4a_{n-3}$\\
$r^3-4r^2-r+4=0$\\
$r_1=1$\quad$r_2=-1$\quad$r_3=4$
\end{center}
So our expression will be of the from $a_n=c_11^n+c_2(-1)^n+c_34^n$ where $c_i$ is coefficient.\\
Now we can use $a_0,a_1 and a_2$ to find the coefficients $c_i$.
\begin{center}
$a_0=4=c_1+c_2+c_3$\\
$a_1=8=c_1-c_2+4c_3$\\
$a_2=34=c_1+c_2+16c_3$\\
$c_1=1$\quad $c_2=1$\quad$c_3=2$
\end{center}
So finally, our solution will be:
\begin{center}
$a_n=1+(-1)^n+2\cdot 4^n$
\end{center}

\section*{Answer 6}
By it's definition, the growth function of formula $a_n=\binom{10}{n+1}$ is:
\begin{center}
$G(x)=\sum_{i=0}^{\infty}\binom{10}{k+1}x^k$
\end{center}
Now we will find a closed form expression for $G(x)$. We will use the binomial expression for this.\\
First note that since when $k>n$ $\binom{n}{k}=0$, we can see $a_n=0$ for $n>9$. This means our generating function's value will be equal to:
\begin{center}
$G(x)=\sum_{i=0}^{\infty}\binom{10}{k+1}x^k=\sum_{i=0}^{9}\binom{10}{k+1}x^k$\\
$= \binom{10}{1}x^0+\binom{10}{2}x^1+\cdots +\binom{10}{10}x^9$
\end{center}
Now, we know the definition of binomial expression to be: (refer to chapter 6.4 theorem 1 for proof)
\begin{center}
$(x+y)^n=\binom{n}{0}x^0y^n+\binom{n}{1}x^1y^{n-1}+\cdots +\binom{n}{n}x^ny^0$
\end{center}
We can make this expression look like our $G(x)$ by taking $y=1$ and $n=10$:
\begin{center}
$(x+1)^{10}=\binom{10}{0}x^0+\binom{10}{1}x^1+\cdots +\binom{10}{10}x^{10}$
\end{center}
Almost there, but we also realize that in $G(x)$ we don't have a term with $\binom{10}{0}$ and unlike the above expression, the power of each $x$ is one more than the second argument of the combination function. To make both expressions equal, we can subtract the first term(which is just $1$) from the above expression and divide the whole thing by $x$.
\begin{center}
$\cfrac{(x+1)^{10}-1}{x}=\cfrac{\binom{10}{0}x^0+\binom{10}{1}x^1+\cdots +\binom{10}{10}x^{10}-1}{x}$\\
$=\binom{10}{1}x^0+\cdots +\binom{10}{10}x^9=G(x)$
\end{center}
Ergo, we found the closed form expression of our growth function to be $\cfrac{(x+1)^10-1}{x}$.

\end{document}

​

