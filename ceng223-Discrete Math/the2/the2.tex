\documentclass[12pt]{article}
\usepackage[utf8]{inputenc}
\usepackage{float}
\usepackage{amsmath}


\usepackage[hmargin=3cm,vmargin=6.0cm]{geometry}
%\topmargin=0cm
\topmargin=-2cm
\addtolength{\textheight}{6.5cm}
\addtolength{\textwidth}{2.0cm}
%\setlength{\leftmargin}{-5cm}
\setlength{\oddsidemargin}{0.0cm}
\setlength{\evensidemargin}{0.0cm}

%misc libraries goes here
\usepackage{amsmath}
\usepackage{amssymb}
\begin{document}




\section*{Student Information } 
%Write your full name and id number between the colon and newline
%Put one empty space character after colon and before newline
Full Name : Deniz Rasim Uluğ \\
Id Number : 2172088 \\

% Write your answers below the section tags
\section*{Answer 1}
a.\\
Let's start from the right hand-side. It is stated that $f_k^{-1}(A_k)=\left\{{x\in E| f_k(x)\in A_k}\right\}$. Therefore we can write right hand-side as follows,
\begin{center}
$\bigcap\limits_{k=1}^n f_k^{-1}(A_k)$\\
$=f^{-1}_1(A_1) \cap f^{-1}_2(A_2)\cap ...\cap f^{-1}_n(A_n)$\\
$=\left\{{x\in E|f_1(x)\in A_1\wedge f_2(x)\in A_2\wedge ...\wedge f_n(x)\in A_n}\right\}$\\
$=\left\{{x\in E|x_1\in A_1\wedge x_2\in A_2\wedge ...\wedge x_n\in A_n}\right\}$\\
for $x=(x_1,x_2,x_3,...,x_n)$
\end{center}
This means that any $i'th$ element of any tuple $x\in E$ should also be an element of $A_i$. The set we formulate in such a way is obviously the definition of cartesian product
\begin{center}
$A_1\times A_2\times ...\times A_n$
$=\prod\limits_{k=1}^n A_k$
\end{center}
Which forms a set of tuples s.t. $x=(x_1,x_2,...,x_n)$ and $x_i\in A_i$.\\
b.\\
Function $f_2$ is not 1-to-1 if $\exists i\: |E_i|\geq 2,\: i=1,3,4,5,...,n$\\\\
The reason is, in such a case we can easily find two different tuples $x_a,x_b\in E$ s.t. their second elements are both $x_2\in E_2$ but their $i'th$ elements are different therefore we would have such a case that
\begin{center}
$f_2(x_a)=x_2$ and $f_2(x_b)=x_2$ and $ x_a\neq x_b$
\end{center}
Therefore $f_2$ is not for the mentioned case 1-to-1.\\
On the other hand, if $\forall i\: |E_i|<2$, in cartesian product there will only be one tuple s.t. $x\in E and |E|=1$ which in that case obviously
\begin{center}
$f_2(x)=x_2$
\end{center}
And the function is 1-to-1.\\\\
The case for all sets $E_i$ are empty the proof is trivial and $f_2$ is an empty function.\\
c.\\
For $f_1$ to be on to, it's co-domain should equal to it's range. Range of $f_1$ is given as $E_1$. Let's choose an arbitrary $x_1\in E_1$ and see if we can find an $x\in E$ s.t. $f(x)=x_1$.\\\\
Since the definition of $E$ is the cartesian product of all tuples $E_i, i=1,2,...,n$, if an element $x_i$ is in $E_i$ we can find at least one tuple $x\in E$ s.t. $i'th$ element of $x$ is $x_i$ and we can say for that tuple $f(x)=x_i$. Applying the same steps for our $x_1$ show that we indeed can find such an $x$.\\\\
Since our $x_1$ was chosen arbitrarily from $f's$ range, it's range should be equal to it's co-domain, ergo function $f_1$ is on to.\\
d.\\
\begin{center}
$\overline{f_K^{-1}(A_k)}=\left\{{x\in E|\neg (f(k)\in A_k )}\right\} $\\
$=\left\{{x\in E|f(k)\not \in A_k}\right\}$\\
$=\left\{{x\in E|f(k)\in \overline{A_k}}\right\}$\\
$=f_k^{-1}(\overline{A_k})$
\end{center}
e.\\The cartesian product
\begin{center}
$A_1\times E_2\times E_3\times ...\times E_n$
\end{center}
will include all tuples from E where first element of the tuple is an element of $A_1$ such that
\begin{center}
$\left\{{ x \in E| x_1\in A_1 \wedge x_i\in E_i}\right\}$\\
$x=(x_1,x_2,...,x_n)$ and $i=2,3,4...n$
\end{center}
We can inverse this set as follows, since by definition $x_i\in E_i$
\begin{center}
$\left\{{ x \in E| x_1\notin A_1 \vee x_i\notin E_i}\right\}$\\
$\left\{{ x \in E| x_1\notin A_1 \vee (false)}\right\}$\\
$\left\{{x\in E| x_1\notin A_1}\right\}$
\end{center}
In other words we are looking for the set of $x\in E$ tuples s.t. $x_1\notin A_1$. Since it is given that $E_k$ is the universal set of $A_k$ we can formulate this as follows
\begin{center}
$(E_1\setminus A_1)\times \prod\limits_{i=2}^n E_i$\\
$\overline{A_1}\times \prod\limits_{i=2}^n E_i$
\end{center}
This concludes our proof.


\section*{Answer 2}

a.\\
To show that f has inverse, we will show it is both 1-to-1 and on to.\\
\underline{1-TO-1}\\
Let's choose two arbitrary $x_1,x_2 \in Z$ s.t.
\begin{center}
$f(x_1)=f(x_2)$
\end{center}
To prove then $x_1=x_2$ is also true, we need to consider three cases.\\
1)$x_1,x_2<0$ 
Then by the definition of the function f we have
\begin{center}
$2|x_1|=2|x_2|$\\$-x_1=-x_2$\\$x_1=x_2$
\end{center}
2)$x_1,x_2\geq 0$ Then by the definition of function f we have
\begin{center}
$2x_1+1=2x_2+1$\\$x_1=x_2$
\end{center}
3)$x_1<0,x_2\geq 0$ Then by the definition of the function f we have
\begin{center}
$2|x_1|=2x_2+1$\\$-x_1-x_2=\dfrac{1}{2}$\\$x_1+x_2=-\dfrac{1}{2}$
\end{center}
Which can't happen because we chose $x_1,x_2$ to be whole numbers so in that case $f(x_1) \neq f(x_2)$, ergo in all cases we conclude $f(x_1)=f(x_2) \to x_1=x_2$ holds true and function f is 1-to-1.\\
\underline{On To}\\
For a function to be on to, it's co-domain should be equal to it's range. Let's try to find the co-domain of each partial function $f_1$ and $f_2$.\\
For $f_1$ let's choose an arbitrary $y\in N^+$ and assume an $x<0$ and $x\in Z$ s.t. $f(x)=y$. Then we have
\begin{center}
$y=-2x$\\$x=\dfrac{y}{-2}$
\end{center}
We see that as long as we choose our $y$ to be divisible by 2, we indeed have such an $x$. We can say co-domain of $f_1$ contains
\begin{center}
$C_1=\left\{{y\in N^+ | \exists k y=2k}\right\}$
\end{center}
Inversely for $f_2$ we do the same steps with an arbitrary $y$ and assumed $x$ to find
\begin{center}
$y=2x+1$\\$x=\dfrac{y-1}{2}$
\end{center}
We see that as long as we choose our $y$ so that $y-1$ is divisible by 2,in other words for $y$ to be odd, we indeed have such an $x$. We can say co-domain of $f_2$ contains
\begin{center}
$C_2=\left\{{y\in N^+|\exists k y=2k+1}\right\}$
\end{center}
Finally, if we take $C_1\cup C_2$ which is the subset of the co-domain of function $f$, it is clear that co-domain is equal to range $N^+$. This concludes our proof.\\
b.\\
It is easy to show from part (a) that $f^{-1}:N^+\to Z$ and

\[f^{-1}(x)= \begin{cases} 
      \dfrac{x}{-2} & x\: is\: even \\
      \dfrac{x-1}{2} & x\:is\:odd\\
   
   \end{cases}
\]
So we find
\begin{center}
$f(26)=\dfrac{26}{-2}=-13$
\end{center}




\section*{Answer 3}

First let's set these to hold true for $\forall n \geq 2$

\begin{align*}
n\geq 2 && \log_2n\geq 1 && n\log_2n\geq 1 && n^2\geq 1
\end{align*}
From these with help of Lemma1 we can conclude the below inequalities
\begin{align*}
\log_2n\leq n^2\log_2n && n\leq n^2\leq n^2\log_2n && n\log_2n\leq n^2\log_2n
\end{align*}
We can also conclude $n\log^2_2n\leq n^2\log_2n$ from $\log_2n\leq n$ with the help of Lemma2.\\
Now by the definition of big-oh let us try to find $k,c$ s.t. $\forall n\geq k\: \exists c\big (f(n)\leq cg(n)\big )$


\begin{center}
$f(n)= 12(\log_2n+n)(n+3n\log_2n)+6n^2 =12\big [ n\log_2n+3n\log_2^2n+n^2+3n^2\log_2n \big ]+6n^2$\\$ \leq 12\big [n^2log_2n+3n^2\log_2n+n^2\log_2n+3n^2\log_2n\big ] +6n^2\log_2n \leq 102n^2\log_2n$
\end{center}

We see for $k=2$ and $c=102$
\begin{center}
$f(n)=O\big (n^2\log_2n\big )=O\big (g(n)\big )$
\end{center}

\underline{Lemma1}\\
Let's choose two $x,y\in N, x>0,y>1$.\\
Let's assume $x>xy$. Since $x>0$, we can divide both sides by x to get $1>y$ which contradicts with our premise. Therefore our assumption is wrong, therefore $x\leq xy$.\\
In other words, a positive number multiplied with another number greater than one will always be less than or equal to himself.\\

\underline{Lemma2}\\
By refering to Chapter 3  Figure 3(Rosem\& Kenneth, Discrete Mathematics and It's Applications,p. 211) we can see for $n\geq 2$, $\log_2n\leq n$. Than from this we can multiple both sides with $n\log_2n$ to get
\begin{center}
$n\log^2_2n\leq n^2\log_2n$
\end{center}


\section*{Answer 4}

Let's assume $E\setminus S$ is countable. \\ \\It is given that $S$ is countable, therefore by Lemma1 $(E\setminus S) \cup S$ is also countable. But then by Lemma2 we see $E \subseteq (E\setminus S) \cup S$ so $E$ is contained in a countable set so it must itself be countable. This contradicts with our premise that $E$ is uncountable, which means our assumption is wrong, ergo $E\setminus S$ is uncountable. \\ \\$\underline{Lemma1}$\\ \\ Let $A$ and $B$ be two countable sets. That means both can be written as \\  $A=\left\{{a_1,a_2,a_3...  }\right\}$ \\ $B=\left\{{b_1,b_2,b_3...  }\right\}$\\ We can take union of these sets such a way that \\ $A\cup B=\left\{ {a_1,b_1,a_2,b_2...}\right\}$\\Which can be listed and is clearly countable. \\ \\ $\underline{Lemma2}$\\ \\ We will try to prove $A\subseteq((A\setminus B) \cup B)$\\ \\ $(A\setminus B)\cup B=\left\{ { x | x\in (A\setminus B) \vee x\in B}\right\} = \left\{{ x| (x\in A \wedge x \not \in B)\vee x \in B} \right\} = \left\{{ x | x\in A \wedge (x\not \in B \vee x \in B)}\right\} = \left\{{ x|x\in A \wedge (TRUE)}\right\}=\left\{{x|x\in A}\right\}=A$

\section*{Answer 5}

a.\\
Assume $n\equiv 1 (mod 3)$. Keeping in mind our Lemma1, then $n+1=1+1=2 (mod 3)$. By multiplying the formulas with each other we conclude the following,
\begin{center} 
$n(n+1)=1\cdot 2=2 (mod3)$
\end{center}
This concludes the first part of our proof. Otherwise, if $n\not \equiv 1(mod3)$ then either $n\equiv 2 (mod 3)$ or $n\equiv 0 (mod 3)$  must be true.\\
For the first case we have $n+1 \equiv 2+1 \equiv 3 \equiv 0(mod3)$ and multiplying we get
\begin{center} 
$n(n+1)=2\cdot 0\equiv 0(mod3)$
\end{center}
For the second case, $n+1\equiv 0+1\equiv 1(mod3)$ and again multiplying with each other
\begin{center}
$n(n+1)\equiv 0\cdot 1\equiv 0(mod3)$
\end{center}
And this concludes our proof.\\
\underline{Lemma1}\\
Refer to Chapter 4 Theorem 5(Rosen\& Kenneth,Discrete Mathematics and It's Applications,p. 242).\\\\
b.\\
$gcd\big (123,277\big )=gcd\big (277,123\big )=gcd\big (123,277(mod\:123)\big )=gcd\big (123,31\big )= gcd\big (31,123(mod\: 31)\big )$\\$=gcd\big (31,30\big )= gcd\big (30,31(mod\:30)\big )=gcd\big (30,1\big )=gcd\big (1,30(mod\:1)\big )=gcd\big (1,0\big )=1$\\\\
c.\\
Let's first see if the first part of the implication holds true. For that, we will try to find the set of possible $p$ values. It is given $p>2$, $p$ is even and $p$ is a prime, so let's write all three sets as follows and take their intersection.
\begin{center}
$P_1 = \left\{ {x\in N| x >2} \right\}=\left\{{3,4,5...} \right\}$
\end{center}
\begin{center}
$P_2=\left\{{x\in N| x\: is\: even}\right\}=\left\{{2,4,6,8,...}\right\}$
\end{center}
\begin{center}
$P_3=\left\{{x\in N| x\: is\: prime}\right\}=\left\{{2,3,5,7,9,...}\right\}$
\end{center}
Let's first try to intersect $P_2$ and $P_3$. $P_2$ implies
\begin{center}
$x\in P_2\to \exists k \: x=2k$
\end{center}
And $P_3$ implies
\begin{center}
$x\in P_3\to \forall x_1,x_2\: \big (x=x_1x_2 \to ((x_1=x \wedge x_2=1) \vee (x_2=x\wedge x_1=1))\big)$
\end{center}
To find an element of intersection $P_2\cap P_3$ we need to find $x$ such that satisfies both equations.\\
Since $x=2k$ that means either $2=1\: k=x$ or $k=1\: 2=x$, obviously only the later is possible. So we can conclude
\begin{center}
$P_2\cap P_3=\left\{{x\in N| x=2}\right\}=\left\{{2}\right\}$
\end{center}
And finally it is obvious that when we intersect the resulting set with $P_1$ we get $\emptyset$ which means we can't find such $p$, ergo the first part of the implication always holds false.\\
We conclude that since $false\to P$ is always true, the given implication is also true.



\end{document}

​

